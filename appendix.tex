\begin{appendices}
\addtocontents{toc}{\protect\renewcommand{\protect\cftchappresnum}{Appendix }}

\chapter{Supplementary Materials for Algorithms and Data}
\section{Data Description}
\subsection{Data BraTS 2012 \& 2013}
We obtained the BRATS 2012 and BRATS 2013 datasets from Professor Raphael Meier at the Institute of Medical Image Analysis for Surgical Technology, University of Berne, Switzerland. These datasets include a diverse array of MRI scans, essential for our research in brain tumor analysis.
\subsection{Data BraTS 2019 \& 2020}
\subsubsection{Data BraTS 2019}
You can download the complete training data, including ground truth segmentation labels and survival data, from:
\href{https://www.cbica.upenn.edu/sbia/Spyridon.Bakas/MICCAI_BraTS/2019/MICCAI_BraTS_2019_Data_Training.zip}{Click Here}
\begin{figure}[H]
  \centering
  \includegraphics[width=0.5\textwidth]{images/frame.png}
  \label{fig:image1}
\end{figure}
You can now also download the independent set of validation scans (without ground truth labels) from:
\href{https://www.cbica.upenn.edu/sbia/Spyridon.Bakas/MICCAI_BraTS/2019/MICCAI_BraTS_2019_Data_Validation.zip}{Click Here}
\begin{figure}[H]
  \centering
  \includegraphics[width=0.5\textwidth]{images/frame (1).png}
  \label{fig:image1}
\end{figure}
\subsubsection{Data BraTS 2020}
You can download the training data, including ground truth segmentation labels and survival data, from the following link:
\href{https://www.cbica.upenn.edu/MICCAI_BraTS2020_TrainingData}{Click Here}
\begin{figure}[H]
  \centering
  \includegraphics[width=0.5\textwidth]{images/frame (3).png}
  \label{fig:image1}
\end{figure}
You can now also download the independent set of validation data (without ground truth labels) from the following link:
\href{https://www.cbica.upenn.edu/MICCAI_BraTS2020_ValidationData}{Click Here}
\begin{figure}[H]
  \centering
  \includegraphics[width=0.5\textwidth]{images/frame (2).png}
  \label{fig:image1}
\end{figure}
\section{Algorithms Used}
\subsection{Algorithm (3.1): NIfTI File Processing Technique}
The objective of this program is to convert NIfTI files, commonly used in medical imaging, into PNG format. In Chapter 3, it is utilized to extract and visualize axial, coronal, and sagittal slices from our data set. This conversion facilitates easier analysis and presentation of the complex medical imaging data (Require Matlab R2017b).
\\
\begin{lstlisting}
% Set the Working Directory
clear; warning off; current = pwd;
path = uigetdir(pwd,'select your working directory');
cd(path);
% Select NIfTI
[file,path] = uigetfile('*.nii');
if isequal(file,0)
   disp('User selected Cancel');
else
   disp(['User selected ', fullfile(path,file)]);

   % Read NIfTI Data and Header Info
   image = niftiread(fullfile(path,file));
   image_info = niftiinfo(fullfile(path,file));
   nifti_array = size(image);
   double = im2double(image);

   % ask user to rotate and by how much
   ask_rotate = input(' Would you like to rotate the orientation? (y/n) ', 's');
   if lower(ask_rotate) == 'y'
       ask_rotate_num = str2double(input('OK. By 90° 180° or 270°? ', 's'));
       if ask_rotate_num == 90 || ask_rotate_num == 180 || ask_rotate_num == 270
           disp('Got it. Your images will be rotated.');
       else
           disp('Sorry, I did not understand that. Quitting...');
           exit;
       end
   elseif lower(ask_rotate) ~= 'n' && lower(ask_rotate) ~= 'y'
       disp('Sorry, I did not understand that. Quitting...');
       exit;
   end
   % If this is a 4D NIfTI
   if length(nifti_array) == 4

       % Create output folder
       mkdir png

       % Get Vols and Slice
       total_volumes = nifti_array(4);
       total_slices = nifti_array(3);

       current_volume = 1;
       disp('Converting NIfTI to png, please wait...')
       % Iterate Through Vol
       while current_volume <= total_volumes
           slice_counter = 0;
            % Iterate Through Slices
            current_slice = 1;
            while current_slice <= total_slices
                % Alternate Slices
                if mod(slice_counter, 1) == 0

                    % Rotate images if selected
                    if lower(ask_rotate) == 'y'
                        if ask_rotate_num == 90
                            data = rot90(mat2gray(double(:,:,current_slice,current_volume)));
                        elseif ask_rotate_num == 180
                            data = rot90(rot90(mat2gray(double(:,:,current_slice,current_volume))));
                        elseif ask_rotate_num == 270
                            data = rot90(rot90(rot90(mat2gray(double(:,:,current_slice,current_volume)))));
                        end
                    elseif lower(ask_rotate) == 'n'
                    disp('OK, I will convert it as it is.');
                    data = mat2gray(double(:,:,current_slice,current_volume));
                    end

                    % Set Filename as per slice and vol info
                    filename = file(1:end-4) + "_t" + sprintf('%03d', current_volume) + "_z" + sprintf('%03d', current_slice) + ".png";

                    % Write Image
                    imwrite(data, char(filename));

                    % If we reached the end of the slices
                    if current_slice == total_slices
                        % But not the end of the volumes
                        if current_volume < total_volumes
                            % Move to the next volume
                            current_volume = current_volume + 1;
                            % Write the image
                            imwrite(data, char(filename));
                        % Else if we reached the end of slice and volume
                        else
                            % Write Image
                            imwrite(data, char(filename));
                            disp('Finished!')
                            return
                        end
                    end

                    % Move Images To Folder
                    movefile(char(filename),'png');

                    % Increment Counters
                    slice_counter = slice_counter + 1;

                    percentage = strcat('Please wait. Converting...', ' ', num2str((current_volume/total_volumes)*100), '% Complete');

                    if ((current_volume/total_volumes)*100) == 100
                        disp('100% Complete! Images successfully converted.');
                    else
                        disp(percentage);
                    end
                end
                current_slice = current_slice + 1;
            end
       current_volume = current_volume + 1;
       end
   % Else if this is a 3D NIfTI
   elseif length(nifti_array) == 3
       % Create output folder
       mkdir png

       % Get Vols and Slice
       total_slices = nifti_array(3);

       disp('Converting NIfTI to png, please wait...')
       slice_counter = 0;
        % Iterate Through Slices
        current_slice = 1;
        while current_slice <= total_slices
            % Alternate Slices
            if mod(slice_counter, 1) == 0

                % Rotate images if selected
                if lower(ask_rotate) == 'y'
                    if ask_rotate_num == 90
                        data = rot90(mat2gray(double(:,:,current_slice)));
                    elseif ask_rotate_num == 180
                        data = rot90(rot90(mat2gray(double(:,:,current_slice))));
                    elseif ask_rotate_num == 270
                        data = rot90(rot90(rot90(mat2gray(double(:,:,current_slice)))));
                    end
                elseif lower(ask_rotate) == 'n'
                disp('OK, I will convert it as it is.');
                data = mat2gray(double(:,:,current_slice));
                end

                % Set Filename as per slice and vol info
                filename = file(1:end-4) + "_z" + sprintf('%03d', current_slice) + ".png";
                % Write Image
                imwrite(data, char(filename));
                % Move Images To Folder
                movefile(char(filename),'png');
                % Increment Counters
                slice_counter = slice_counter + 1;
                percentage = strcat('Please wait. Converting...', ' ', num2str((current_slice/total_slices)*100), '% Complete');
                if ((current_slice/total_slices)*100) == 100
                    disp('100% Complete! Images successfully converted');
                else
                    disp(percentage);
                end
            end
            current_slice  = current_slice  + 1;
        end
   elseif length(nifti_array) ~= 3 || 4
       disp('NIfTI must be 3D or 4D. Please try again.');
   end
end
\end{lstlisting}
\subsection{Algorithm (3.2): MATLAB Implementation for the Off-Axis Optical Scanning Holography for Brain Tumor Detection}
The MATLAB program provided implements off-axis Optical Scanning Holography (OSH) for the detection and visualization of brain tumors. Here is a summary of the program and its key steps:

1. Image Preparation: The program begins by reading a grayscale brain scan image, which is resized to 256x256 pixels for standardization. This step ensures uniformity in processing across different images.

2. Optical Transfer Function (OTF) Creation: An Optical Transfer Function (OTF) is generated for off-axis OSH using specific spatial frequencies and a scaling factor (sigma). The OTF is crucial for modulating the image data in the Fourier domain, facilitating the holographic process.

3. Hologram Recording: The program records the hologram by applying the OTF to the Fourier-transformed image. This step involves the normalization of the Fourier-transformed image, followed by an element-wise multiplication with the OTF, and then an inverse Fourier transform to obtain the hologram.

4. Hologram Reconstruction: Several reconstructions of the hologram are performed, including the real part, imaginary part, and complex part of the hologram. These reconstructions are essential for extracting meaningful information that can be used in the detection of anomalies such as tumors.

5. Visualization of Reconstructed Hologram: The reconstructed hologram is visualized using a mesh plot. This visualization provides an intuitive understanding of the holographic data and aids in identifying significant features within the hologram.

6. Peak Detection: The program specifically extracts the maximum peak of the phase component from the scanned current. This peak is a critical marker in the holographic data, corresponding directly to the location of the brain tumor. This extraction is a pivotal step in the detection process, as it isolates the most significant feature in the hologram that is indicative of tumor presence.

7. Visualization of Detected Peak on Original Image: The detected peak is overlaid on the original brain scan image. This step visually indicates the potential location of the tumor, correlating the holographic data with the actual image.

8. Active Contour Segmentation: The program uses active contour segmentation to refine the detection of the tumor. This process is initiated around the detected peak, creating a circular region as a starting point for segmentation. The active contour algorithm iteratively adjusts this region to segment the tumor precisely.

9. Segmented Tumor Visualization: Finally, the segmented tumor is displayed on the original image. This step provides a clear indication of the tumor's location and extent, demonstrating the effectiveness of the combined holographic and segmentation approach.
\\
\begin{lstlisting}
% Reading the input image file
I = imread('brain-tumor1.jpg', 'jpeg'); % Replace 'brain-tumor1.jpg' with your specific image file
I = I(:, :, 1); % Using only one color channel (grayscale)
I = imresize(I, [256 256]); % Resizing the image to 256x256 pixels

% Creating the Optical Transfer Function (OTF) for off-axis OSH
ROWS = 256;
COLS = 256;
sigma = 2.0; % Scale factor for OTF
ky = -12.8; % Initial spatial frequency for y
OTFosh = zeros(ROWS, COLS);
for r = 1:COLS
    kx = -12.8; % Initial spatial frequency for x
    for c = 1:ROWS
        OTFosh(r, c) = exp(-j * sigma * kx * kx - j * sigma * ky * ky); % Creating the OTF
        kx = kx + .1; % Incrementing the spatial frequency for x
    end
    ky = ky + .1; % Incrementing the spatial frequency for y
end
OTFosh = OTFosh .* (1.0 / max(max(OTFosh))); % Normalizing the OTF

% Recording the hologram in the Fourier domain
FI = fftshift(fft2(I)); % Fourier transform of the image
FI = FI .* (1.0 / max(max(FI))); % Normalizing the Fourier image
FH = FI .* OTFosh; % Applying the OTF to the Fourier image
H = ifft2(FH); % Inverse Fourier transform to get the hologram
H = H .* (1.0 / max(max(H))); % Normalizing the hologram

% Reconstruction of the hologram
FRSINEH = fft2(real(H)) .* conj(OTFosh); % Fourier transform of the real part of the hologram
RSINEH = ifft2(FRSINEH); % Inverse Fourier transform for sine hologram reconstruction
FRCOSINEH = fft2(imag(H)) .* conj(OTFosh); % Fourier transform of the imaginary part of the hologram
RCOSINEH = ifft2(FRCOSINEH); % Inverse Fourier transform for cosine hologram reconstruction
FRCOMPLEXH = fft2(real(H) + j*imag(H)) .* conj(OTFosh); % Fourier transform of the complex hologram
RCOMPLEX = ifft2(FRCOMPLEXH); % Inverse Fourier transform for complex hologram reconstruction
R6 = 1.4 * 256 * abs(RCOMPLEX) / max(max(abs(RCOMPLEX))); % Normalizing and scaling the reconstructed hologram

% Visualizing the reconstructed hologram
[H, L] = size(R6);
[X, YY] = meshgrid(1:L, 1:H);
figure
mesh(X, -YY, R6); % Displaying the reconstructed hologram as a mesh plot
colorbar;
pderot3d ON; % Enabling 3D rotation of the plot

% Detecting the peak in the reconstructed hologram
[T, YHOLO] = find(R6 == max(R6(:))); % Finding the coordinates of the peak
PIC1 = zeros(H, L); % Initializing a matrix for peak visualization
PIC1(T-5:T+5, YHOLO) = 255; % Marking the peak in the matrix
PIC1(T, YHOLO-5:YHOLO+5) = 255;

% Displaying the detected peak on the original image
figure
imshow(I); % Displaying the original image
hold on;
visboundaries(PIC1, 'Color', 'r'); % Visualizing the peak location on the image
% Active contour for tumor segmentation
% Initializing the segmentation around the detected peak
centreX = T + abs(T - T2);
centreY = Y + abs(Y - Y2);
[x, y] = meshgrid(1:H, 1:L);
discseg = real(hypot(x - centreY, y - centreX) < 5); % Creating a circular region around the peak
Maxeiterations = 200; % Maximum number of iterations for active contour
tumorseg = activecontour(I, discseg, Maxeiterations, 'Chan-Vese'); % Active contour segmentation

% Displaying the segmented tumor
figure
imshow(I); % Displaying the original image
hold on;
title('\color{red} Tumor segmented after Active Contour Evolution');
visboundaries(tumorseg, 'Color', 'r'); % Visualizing the segmented tumor
% End of the MATLAB program
\end{lstlisting}

\subsection{Algorithm (3.3): In-Line Optical Scanning Holography for Brain Tumor Detection}
This MATLAB program executes brain tumor detection using in-line Optical Scanning Holography (OSH). It starts by reading a brain scan, converting it to grayscale, and standardizing its size. The program then generates a specialized Optical Transfer Function (OTF) for in-line OSH, crucial for holography. Next, it records the hologram by applying this OTF to the image's Fourier transform. The hologram is reconstructed, addressing the twin-image issue characteristic of in-line methods. Post-processing techniques are applied to enhance the reconstructed image for clearer tumor detection. The program identifies the tumor's location and overlays this information on the original scan, providing a direct visual correlation of the tumor in the brain scan. This approach integrates advanced holographic imaging with computational processing for effective tumor visualization.
\\
\begin{lstlisting}
% Step 1: Reading and Preprocessing the Input Image
inputImageFile = 'BrainScan.jpg'; % Replace with your specific image file
I = imread(inputImageFile);
I = rgb2gray(I); % Converting to grayscale if it's a color image
I = imresize(I, [256 256]); % Resizing the image to 256x256 pixels
figure;
imshow(I);
title('Original Brain Scan Image');

% Step 2: Generating the Optical Transfer Function (OTF) for In-Line OSH
% Define parameters for OTF creation
sigma = 2.0; % Scale factor for OTF
ROWS = 256; COLS = 256;
OTF = createInLineOTF(sigma, ROWS, COLS); % Function to create in-line OTF

% Step 3: Recording the Hologram
FI = fftshift(fft2(I)); % Fourier transform of the image
FI = normalize(FI); % Normalizing the Fourier-transformed image
FH = FI .* OTF; % Applying the OTF to the Fourier image
H = ifft2(FH); % Inverse Fourier transform to get the hologram
H = normalize(H); % Normalizing the hologram
figure;
imshow(log(1 + abs(H)), []);
title('Recorded Hologram');

% Step 4: Reconstructing the Hologram
% Reconstruct the hologram while addressing the twin-image problem
R = reconstructHologram(H, OTF); % Function to reconstruct the hologram
figure;
imshow(log(1 + abs(R)), []);
title('Reconstructed Hologram');

% Step 5: Post-Processing the Reconstructed Image
% Apply image processing techniques like filtering, enhancement, etc.
processedImage = processReconstructedImage(R);
figure;
imshow(processedImage);
title('Processed Reconstructed Image');

% Step 6: Identifying Tumor Location
% Detect the tumor location using image processing techniques
[tumorLocation, tumorMask] = identifyTumor(processedImage);
figure;
imshow(tumorMask);
title('Tumor Location');

% Step 7: Overlaying Tumor Location on Original Image
figure, imshow(I); % Displaying the original image
hold on;
visboundaries(tumorMask, 'Color', 'r'); % Visualizing the tumor location
title('Tumor Location on Original Image');
\end{lstlisting}
\subsection{Algorithm (4.1): MATLAB Implementation for Brain Tumor Detection via Hybrid Method: the Optical Scanning Holography and the Optical Correlation}
This annex presents a MATLAB program that demonstrates a hybrid method for brain tumor detection, combining Optical Scanning Holography (OSH) with Optical Correlation. The process starts with the acquisition of a brain scan image, which is resized for uniform analysis. A crucial step is the creation of an Optical Transfer Function (OTF), pivotal in generating a holographic image of the brain. This OTF, defined by specific spatial frequencies and sigma values, is instrumental in extracting the peak of the phase component from the hologram a vital phase in OSH that reveals the brain's intricate structural details. The program carefully examines the hologram's sine and cosine components, each providing unique insights into the holographic data. Following this, the sine component of the Fourier-Zone Plate (FZP) hologram is reconstructed, an essential step in retrieving meaningful information from the hologram. Notably, this method meticulously extracts the peak of the phase component, a significant indicator of the brain's anatomical features. The program's hybrid approach is further highlighted in its application of Optical Correlation, where it adeptly identifies the peak of the correlation distribution. This correlation peak is a critical marker, signifying the potential location of the tumor. The combination of these advanced techniques enables the program to integrate active contour theory, which refines the tumor's segmentation with high precision, based on the initial holographic and correlation detections. The program culminates in displaying the tumor's location on the original brain scan, alongside a segmented image of the tumor. This blend of OSH and Optical Correlation, enriched with computational analytics, marks a significant advancement in medical imaging, particularly in the accurate detection and detailed segmentation of brain tumors.

1. Image Acquisition and Preparation: The program begins by acquiring a brain scan image. This image is resized to a uniform scale for consistent analysis throughout the process.

2. Creation of Optical Transfer Function (OTF): A crucial step involves generating an Optical Transfer Function (OTF). This OTF, defined by specific spatial frequencies and a sigma value, is pivotal in creating a holographic image of the brain.

3. Extraction of Phase Component Peak: The OTF is instrumental in extracting the peak of the phase component from the hologram. This step is vital in Optical Scanning Holography (OSH), as it reveals detailed structural information about the brain.

4. Examination of Hologram Components: The program analyzes the sine and cosine components of the hologram. Each component provides unique insights into the holographic data, contributing to a comprehensive understanding of the brain’s structure.

5. Reconstruction of Sine Component of FZP Hologram: An essential part of the process is the reconstruction of the sine component of the Fourier-Zone Plate (FZP) hologram. This reconstruction is critical for retrieving meaningful information from the hologram.

6. Detection of Anatomical Features: The program meticulously extracts the peak of the phase component, a significant indicator of the brain's anatomical features, possibly pointing to the presence of a tumor.

7. Application of Optical Correlation: The hybrid approach of the program is further emphasized through the application of Optical Correlation. This technique adeptly identifies the peak of the correlation distribution, a crucial marker for the potential location of the tumor.

8. Integration of Active Contour Theory: The program incorporates active contour theory to refine tumor segmentation. This method utilizes the initial holographic and correlation detections to segment the tumor with high precision.

9. Visualization of Tumor Location and Segmentation: The final step displays the tumor's location on the original brain scan, along with a segmented image of the tumor. This visualization demonstrates the effectiveness of the combined techniques
\begin{lstlisting}
% Optical Scanning Holography and Optical Correlation for Brain Tumor Detection

% Clearing the workspace and closing all figures
clear all;
close all;

% Reading the input image
inputImageFile = 'Tumour31_T1.png'; % Replace with your image file
I = imread(inputImageFile, 'png');
I = I(:, :, 1); % Considering only one channel if it's a color image
I = imresize(I, [256 256]); % Resizing image to 256x256 pixels
figure(1) % Displaying the input image
colormap(gray(255));
image(I)
title('Original Image')

% Creating OTFosh with SIGMA=z/2*k0 (Equation 3.5-1a)
ROWS = 256;
COLS = 256;
sigma = 2.0; % Scale factor
ky = -12.8;
OTFosh = zeros(ROWS, COLS);
for r = 1:COLS
    kx = -12.8;
    for c = 1:ROWS
        OTFosh(r, c) = exp(-j * sigma * kx * kx - j * sigma * ky * ky);
        kx = kx + .1;
    end
    ky = ky + .1;
end
max1 = max(OTFosh);
max2 = max(max1);
scale = 1.0 / max2;
OTFosh = OTFosh .* scale;

% Recording hologram
FI = fft2(I);
FI = fftshift(FI);
max1 = max(FI);
max2 = max(max1);
scale = 1.0 / max2;
FI = FI .* scale;
FH = FI .* OTFosh; % FH is the recorded hologram in the Fourier domain
H = ifft2(FH);
max1 = max(H);
max2 = max(max1);
scale = 1.0 / max2;
H = H .* scale;
figure(2)
colormap(gray(255));
image(2.5 * real(256 * H));
title('Sine-FZP hologram')
figure(3)
colormap(gray(255));
image(2.5 * imag(256 * H));
title('Cosine-FZP hologram')

% Reconstruction of sine-FZP hologram
figure(4)
colormap(gray(255))
H = ifft2(FH);
FRSINEH = fft2(real(H)) .* conj(OTFosh); % Equation 2.5-10
RSINEH = ifft2(FRSINEH);
image(256 * abs(RSINEH) / max(max(abs(RSINEH))))
title('Reconstruction of sine-FZP hologram')

% Correlation peak detection and visualization
[H, L] = size(RSINEH);
[X, YY] = meshgrid(1:L, 1:H);
figure(5)
mesh(X, -YY, RSINEH); colorbar
title('Correlation Peak without Noise');

% Hybrid reconstruction with active contour theory
plan1 = imresize(plan, [256 256]);
Hybrid = R5 + plan1;
figure(6)
mesh(X, -YY, Hybrid); colorbar
title('Hybrid Reconstruction');

% Finding the peak in correlation
[T, YHOLO] = find(RSINEH == max(RSINEH(:)));
PIC1 = zeros(H, L);
PIC1(T-5:T+5, YHOLO) = 255;
PIC1(T, YHOLO-5:YHOLO+5) = 255;

% Displaying detected tumor location on the original image
figure(7)
imshow(I); hold on
visboundaries(PIC1, 'Color', 'r'); % HOLO

% Active contour evolution for tumor segmentation
centreX = T + abs(T - T2);
centreY = Y + abs(Y - Y2);
[x, y] = meshgrid(1:H, 1:L);
discseg = real(hypot(x - centreY, y - centreX) < 5);
Maxeiterations = 120;
tumorseg = activecontour(I, discseg, Maxeiterations, 'Chan-Vese');

% Displaying tumor segmentation results
figure(8);
imshow(I); title('\color{red} Tumor segmented after Active Contour Evolution'); hold on
visboundaries(tumorseg, 'Color', 'r');

% End of the program

\end{lstlisting}

\chapter{Experimental Procedure}

\chapter{Questionnaires}
\label{ap:questionnaires}

% demonstration of how to insert PDF appendix with page numbers

\includepdf[pages=1, scale=0.8, offset=0.3in 0, pagecommand=\section{User Experience Questionnaire}]{appendices/PDFs/ueq.pdf}
\label{ap:ueq}
\includepdf[pages=2, scale=0.8, offset=0.3in 0, pagecommand={}]{appendices/PDFs/ueq.pdf}

\end{appendices}


